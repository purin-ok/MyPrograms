\documentclass[titlepage,a4paper]{jsarticle}
\usepackage{../sty/import}% 各種パッケージインポート
\usepackage{../sty/title}% タイトルページの変更

\renewcommand*{\thesubsection}{\arabic{subsection}}
%% タイトルページの変数
% レポートタイトル
\title{信号処理レポート課題}
% 提出日
\expdate{\today}
% 科目名
\subject{信号処理}
% 分野
\class{情報経営システム工学分野}
% 学年
\grade{B3}
% 学籍番号
\mynumber{24336488}
% 記述者
\author{本間三暉}
% グループ名 % もし班があるやつならtitle_team.styを入れる
% \team{10}
\begin{document}
% titleページ作成
\maketitle
\section{フーリエ級数展開、フーリエ変換、離散時間フーリエ変換、離散フーリエ変換の共通点と相違点について説明せよ。}
\section*{共通点}
\subsection{周波数領域での解析}
時間領域の信号を周波数領域で表現することを目的としている.これにより,信号の周波数成分を分析することができる.
\subsection{基底関数}
正弦波や余弦波(フーリエ級数の場合)や複素指数関数(フーリエ変換の場合)を基底関数として使用し信号を展開する.
\subsection{線形性}
線形性を持ち,重ね合わせの原理が適用される.
\subsection{逆変換}
周波数領域の表現を時間領域に戻すための逆変換がある.
\setcounter{subsection}{0}
\section*{相違点}
フーリエ級数展開、フーリエ変換、離散時間フーリエ変換、離散フーリエ変換の相違点として,対象信号,表現形式,適用範囲,数式表現の4つを挙げる.
\subsection{フーリエ級数展開}
対象信号: 離散時間信号を対象とする(サンプリングされた信号).

表現形式: 離散時間信号を連続的な周波数成分として表現する.

適用範囲: 無限長の離散時間信号に適用される.

数式表現: $f(t) = a_0 + \sum_{n=1}^{\infty} \left( a_n \cos\left( \frac{2\pi nt}{T} \right) + b_n \sin\left( \frac{2\pi nt}{T} \right) \right)$
\subsection{フーリエ変換}
対象信号: 非周期信号や無限長の信号を対象とする.

表現形式: 信号を連続的な周波数成分として表現する.

適用範囲: 任意の関数や時間連続の信号に適用される.

数式表現: $F(\omega) = \int_{-\infty}^{\infty} f(t) e^{-i\omega t} \, dt$
\subsection{離散時間フーリエ変換}
対象信号: 離散時間信号を対象とする(サンプリングされた信号).

表現形式: 離散時間信号を連続的な周波数成分として表現する.

適用範囲: 無限長の離散時間信号に適用される.

数式表現: $X(\omega) = \sum_{n=-\infty}^{\infty} x[n] e^{-i\omega n}$
\subsection{離散フーリエ変換}
対象信号: 離散時間信号の有限長区間を対象とする.

表現形式: 信号を有限個の周波数成分として表現する.

適用範囲: 信号処理の実際の応用において、有限のデータに対して適用される.

数式表現: $X[k] = \sum_{n=0}^{N-1} x[n] e^{-i \frac{2\pi k n}{N}}$
\section{本講義の範囲では、近似誤差を測る際に2乗誤差がよく採用されていた。2乗誤差が採用される理由について説明せよ。}
二乗誤差が信号処理,統計学,機械学習など多くの分野で広く採用される理由について説明する.
二乗誤差は,観測値と予測値(または実際の値)の差の二乗を取り,それらを合計したものである.
二乗誤差を最小化することで,モデルや予測器の性能を最適化することができる.
\section*{二乗誤差の定義}
観測値$y_i$と予測値$\hat{y_i}$があるとき,$n$をデータ数として二乗誤差(MSE: Mean Squared Error)式\eqref{定義}で定義される.
\begin{align}
  MSE(y_{i},\hat{y_{i}}) & = \frac{1}{n}\sum_{i=1}^{n}\left(y_{i}-\hat{y_{i}}\right)\label{定義}
\end{align}
\section*{二乗誤差が使われる理由}
二乗誤差が扱われる理由として,以下5つの理由が挙げられる.
\subsection{数学的な扱いやすさ}
二乗誤差は微分可能なので,最小化問題に対して解析的な解法が使いやすい.
例えば,勾配降下法や最小二乗法などが使える.
線形回帰では,二乗誤差を最小化するために単純な線形代数の操作で解を求めることができる.

$MSE(y_{i},\hat{y_{i}})$を$\hat{y_i}$で微分したものを式\eqref{微分}に示す.
\begin{align}
  \frac{\partial}{\partial x} MSE(y_{i},\hat{y_{i}}) & = \frac{2}{n}\left(\hat{y_i} - y_i\right)\label{微分}
\end{align}
\subsection{大きな誤差のペナルティ}
二乗誤差は誤差を二乗するため,大きな誤差に対して非常に大きなペナルティを課すことができる.
これにより,モデルは大きな誤差を避け,全体の予測精度を向上させることができる.
例えば,観測値と予測値の差が大きくなると,その二乗はさらに大きくなり,モデルはその大きな誤差を修正しようとする.
\subsection{最適化の一意性}
二乗誤差を最小化する問題は,通常一意の解を持つ.
つまり,最適なパラメータを見つけることができる.
これは他の誤差指標(例えば絶対値誤差など)には必ずしも当てはまらない特性である.
二乗誤差の凸性により,局所最小値が唯一のグローバル最小値であることが保証されるため,最適化プロセスが安定する.
\subsection{確率論的な解釈}
二乗誤差を最小化することは,ガウス分布に従う誤差を仮定した場合の最尤推定(Maximum Likelihood Estimation, MLE)と一致する.
観測値と予測値の誤差が正規分布に従うと仮定すると,二乗誤差を最小化することが最も確からしいパラメータを推定することになる.
\subsection{計算効率}
二乗誤差の計算は簡単で,コンピュータ上で効率的に計算することができる.
大規模なデータセットやリアルタイムのシステムにおいても計算が現実的である.

\section*{まとめ}
このような理由から,信号処理だけでなく統計学や機械学習などの幅広い分野で利用される評価基準になっていると考えられる.
\end{document}
