\documentclass[titlepage,a4paper]{jsarticle}
\usepackage{../sty/import}% 各種パッケージインポート
\usepackage{../sty/title}% タイトルページの変更

%% タイトルページの変数
% レポートタイトル
\title{信号処理レポート課題}
% 提出日
\expdate{\today}
% 科目名
\subject{信号処理}
% 分野
\class{情報経営システム工学分野}
% 学年
\grade{B3}
% 学籍番号
\mynumber{24336488}
% 記述者
\author{本間三暉}
% グループ名 % もし班があるやつならtitle_team.styを入れる
% \team{10}
\begin{document}
% titleページ作成
\maketitle
\section{フーリエ級数展開、フーリエ変換、離散時間フーリエ変換、離散フーリエ変換の共通点と相違点について説明せよ。}

\section{本講義の範囲では、近似誤差を測る際に2乗誤差がよく採用されていた。2乗誤差が採用される理由について説明せよ。}

\section*{参考文献}
\begin{enumerate}
  \item 令和6年度信号処理テキスト
  \item
\end{enumerate}
\end{document}
