\documentclass[titlepage,a4paper]{jsarticle}
\usepackage{../../../sty/import}% 各種パッケージインポート
\usepackage{../../../sty/title}% タイトルページの変更
\renewcommand{\thesection}{課題\arabic{section}}
\setcounter{section}{6}
% 右上側に名前,学籍番号,提出日の記述が必要なので,急遽付け足し
\usepackage{fancyhdr}
\pagestyle{fancy}
  \lhead{10/19情報社会と著作権課題}
  \rhead{
    \begin{tabular}[t]{>{\bfseries}l@{\hspace{0.5\zw}:\hspace{0.5\zw}}l}
      氏名   & 本間三暉\\
      学籍番号 & 24336488\\
      提出日  & \today
    \end{tabular}
  }
%% タイトルページの変数
% レポートタイトル
\title{情報と著作権課題7〜12}
% 提出日
\expdate{\today}
% 科目名
\subject{情報社会と著作権}
% 分野
\class{情報経営システム工学分野}
% 学年
\grade{B3}
% 学籍番号
\mynumber{24336488}
% 記述者
\author{本間三暉}
%
% 記載例:
%\coauthor{%
% 学籍番号:24567321 & 氏名:吉田 富美男 \
% 学籍番号:12345678 & 氏名:安藤 雅洋 \
% 学籍番号:13579234 & 氏名:雲居 玄道 \
%%

\begin{document}
% titleページ作成
\maketitle
\section{「公衆」の定義を調べて回答してください}
社会一般の人々。世間の人々。大衆。社会学的には、新聞、ラジオなどを媒介とする間接的な接触によって成立し、一定の意見(世論)を形成しうる集団をいう。
同一の関心で同一時に、同一の場所に集合し、類似した反応を一時的に表示することによって成立する群集とは区別される。
また、主体的合理的に判断して行動するものとされ、受動的、非合理的に判断し行動する「大衆」と区別されることがある。\cite{kousyuu}
\section{Aさんの著作物がBさんが翻訳した場合,この翻訳物をCさんが複製するには誰の了解が必要か,理由をつけて回答してください.
  なお,Cさんの複製は私的利用ではないものとする.}
CさんがBさんの翻訳したAさんの著作物を複製するにはAさんとBさん双方の許諾が必要である.
まず,Aさんは原著作物の著作権を保有しており,
Aさんの作品を他の形式(ここでは翻訳)で使用するためには「翻訳権」を有しているためAさんの許可が必要となる\cite{TMJ}\cite{27}.

一方,BさんはAさんの著作物を翻訳し,翻訳物に対する「翻訳著作権」を持つため,Bさんの許可も求められる\cite{OCiETe}.

このように,翻訳物は二次的著作物として扱われ,複製や公開などに関しては原著作者および翻訳者の双方の権利が保護されている\cite{JOHO}.

\section{著作権(財産権)の保護期間の計算方法を調べ回答してください.}
著作権(財産権)の保護期間は,著作物の種類や著作者の公開状況により異なるが,日本では原則として著作者の死亡から70年間とされている.
具体的な計算方法は,著作者の死亡年や著作物の公表年の「翌年1月1日」から起算され,そこから70年間の保護が適用される\cite{neta}\cite{qa}.

例えば,著作者が2015年に亡くなった場合,保護期間は2016年1月1日から数えて70年間,つまり2085年12月31日までである\cite{NAKAGAKI}.
無名または変名の著作物や団体名義の著作物の場合,著作物が公表された年の翌年から70年間が保護期間とされており,一般の著作物とは異なる基準で計算される\cite{q2}.

このように,著作権の保護期間は著作者の死亡日や著作物の公表日を基準に翌年から起算されるため,計算方法が明確に定められている.

\section{実演とはなにか,回答してください.また,実演家人格権の内容について回答してください.}
\subsection{実演とは}
実演とは,著作物や芸能的な表現を演じる・舞う・演奏するなどの形で表現する行為を指す.
たとえば,俳優が舞台で演技をしたり,歌手がステージで歌を披露したりする行為がこれに該当する.
著作物以外のものであっても,芸能的な性質を有するものであれば実演に含まれる\cite{q3}.

\subsection{実演家人格権の内容}
実演家人格権は,実演家が自分の実演に対して有する人格的利益を保護するための権利であり著作隣接権の一種である.
この権利には「氏名表示権」と「同一性保持権」の2つが含まれる.
氏名表示権とは,実演の公衆への提供に際し,実演家が自身の名前(本名や芸名)を表示するかどうかを決める権利である.
同一性保持権は,実演家の意に反して実演内容を改変されない権利で,実演家の名誉や声望を守る目的がある\cite{q3_1}.

\section{実演加盟についての「変名」とはなにか,その定義と具体例を調べ,回答してください.}
\subsection{変名とは}
変名は,実演家や著作者が本名以外の名前で活動する際に用いる別名のことであり,芸名やペンネームなどが典型的な例である.
これにより,著作者はプライバシーを保護しつつ創作活動を行うことが可能になる\cite{henmei}.

\subsection{具体例}
作家「夏目漱石」は本名を「夏目金之助」として知られ,変名の「夏目漱石」を用いて執筆活動を行った.
また,ジャニーズ所属のアーティストなども芸名を使って活動する例が多く,実名を伏せることが一般的である\cite{q4_1}.

なお,変名を使用する場合も著作隣接権としての権利は認められ著作権の保護期間も適用されるが,著作者の死亡日が不明な場合は公表後の期間を基準に計算される\cite{q4_2}.
\section{あなたの好きな実演家の名前を挙げ,その名前が実名,変名のいずれであるか,
  またその実演家はどのような実演をしている者か,簡単に説明してください.}
花澤香菜さんは日本の人気声優であり,主にアニメやナレーションを中心に活動している実演家である.
名前の「花澤香菜」は芸名ではなく実名である.

彼女は『化物語』の「千石撫子」役や『PSYCHO-PASS サイコパス』の「常守朱」役などで知られており,キャラクターの性格に応じた幅広い演技力が高く評価されている.

また,声優業に加え音楽活動も行っており,歌手としても活躍している.
彼女の歌声はファンに支持されておりライブ活動などを通じても多くのファンを魅了している.

\newpage
% 参考文献
\begin{thebibliography}{99}
  \bibitem{kousyuu} 公衆(コウシュウ)とは? 意味や使い方 - コトバンク\url{https://kotobank.jp/word/%E5%85%AC%E8%A1%86-62229}
  \bibitem{TMJ} 翻訳物の著作権について知ろう!著作権の所在とトラブル回避|翻訳会社・通訳会社のTMJ JAPAN\url{https://tmjjapan.co.jp/topics/4602/}
  \bibitem{27} 翻訳権、編曲権、変形権、翻案権(著作権法27条) | 顧問弁護士なら篠原・森法律事務所
  \\\url{https://shinohara-law.com/blog/2016/04/21/}
  \bibitem{OCiETe} 翻訳物の著作権について解説!トラブルを回避するためのポイントを解説|OCiETe通訳・翻訳コラム\url{https://ociete.jp/column/translation-knowledge/5257}
  \bibitem{JOHO} 翻訳した原稿の著作権はどうなるの? - 英語翻訳サービス,多言語翻訳を料金,品質で選ぶならJOHO\url{https://www.joho-translation.com/news/5158/}
  \bibitem{neta} 著作権の保護期間 | 著作権のネタ帳\url{https://copyright-topics.jp/basic/protection_period/}
  \bibitem{qa} 保護期間の計算方法について教えてください.|CopyrightQ\&A著作権なるほど質問箱\url{https://copyright-qa.azurewebsites.net/Qa/0000186}
  \bibitem{NAKAGAKI} 著作権法57条 保護期間の計算方法|NAKAGAKI\_IP
  \\\url{https://note.com/nkgk/n/n4e02033cbaa7}
  \bibitem{q2} 著作権の保護期間は何年?いつまで?著作物の態様別に弁護士がわかりやすく解説 | Authense法律事務所\url{https://www.authense.jp/komon/blog/ip/2802/}
  \bibitem{q3} 実演家人格権 - 用語辞典\url{https://www.cpra.jp/glossary/sa/post_3.html}
  \bibitem{q3_1} 実演家の権利 | 著作権のネタ帳\url{https://copyright-topics.jp/basic/jitsuenka_right/}
  \bibitem{henmei} 「実演家」とは?俳優やアーティストのこと…俳優やアーティストの権利は?「著作隣接権」 | 著作権&著作隣接権まとめノート…ときどきエンタメ日誌
  \\\url{https://learning-artistsrights.com/perfomer-neighboringrights/}
  \bibitem{q4_1} 変名 - Wikipedia\url{https://ja.wikipedia.org/wiki/%E5%A4%89%E5%90%8D}
  \bibitem{q4_2} 実演家とは - 芸団協CPRA 実演家著作隣接権センター\url{https://www.cpra.jp/performers/about/}
\end{thebibliography}

\end{document}