\documentclass[titlepage,a4paper]{jsarticle}
\usepackage{../../../sty/import}% 各種パッケージインポート
\usepackage{../../../sty/title}% タイトルページの変更
\renewcommand{\thesection}{課題\arabic{section}}
\setcounter{section}{12}
% 右上側に名前,学籍番号,提出日の記述が必要なので,急遽付け足し
\usepackage{fancyhdr}
\pagestyle{fancy}
  \lhead{10/26情報社会と著作権課題}
  \rhead{
    \begin{tabular}[t]{>{\bfseries}l@{\hspace{0.5\zw}:\hspace{0.5\zw}}l}
      氏名   & 本間三暉\\
      学籍番号 & 24336488\\
      提出日  & \today
    \end{tabular}
  }
%% タイトルページの変数
% レポートタイトル
\title{情報と著作権課題13〜18}
% 提出日
\expdate{\today}
% 科目名
\subject{情報社会と著作権}
% 分野
\class{情報経営システム工学分野}
% 学年
\grade{B3}
% 学籍番号
\mynumber{24336488}
% 記述者
\author{本間三暉}

\begin{document}
% titleページ作成
\maketitle
\section{私的利用のための複製に関する「家庭内など限られた範囲内」という要件について,
  数人のグループがこの要件に当てはまる場合と当てはまらない場合を調べて回答してください.}%%%
家庭内など限られた範囲内での使用に限られます.
したがって,家庭内であっても,家族以外の者が含まれる場合や,家族であっても家庭外の者が含まれる場合は,私的使用とは認められません\cite{R6}.

\section{言語の著作物を引用し,それに対する自分の批評を述べてください.引用部分の明確化,主従関係,必要最低限度の分量,出所の明示に留意すること.}
榎宮祐のライトノベル『ノーゲーム・ノーライフ』では,ゲームという架空の設定を通して現実の競争社会や人間関係の本質を見事に浮き彫りにしている.
以下の一文は,特にその特徴が顕著に表れている.

\subsection{引用}
「ゲームってのは、究極的には2つしか取れる行動がない。戦術的行動か、対処的行動、主導権どっちが握るか」\cite{no_game_no_life}

\subsection{批評}
この一文は,ゲームの勝敗における戦略の要素を的確に表している.
戦術的行動とは,相手の一歩先を読み,先手を取って状況を支配するアプローチであり,一方で対処的行動とは,相手の出方を待ち,
その動きに合わせて対応するアプローチを指している.
この「主導権をどちらが握るか」という問いかけは,単なるゲームの戦略を超え,人生やビジネスにおける競争の本質に迫っている.
空が常に戦術的行動を重視し,戦いの主導権を握る姿勢は,勝利を目指して考え抜かれた計算と冷静な判断を象徴している.

この発言は,単に彼ら「 」の戦略的思考を表すだけでなく,読者に対しても「行動の主導権を握ることの重要性」という普遍的なテーマを提起している.
現実においても,何事においても主導権を握るか否かでその後の展開が大きく変わる場面は多く,榎宮祐がこの作品を通して投げかけるのは,
読者自身が「どちらの行動を取るのか」を意識させることである.
こうした作品のメッセージ性が『ノーゲーム・ノーライフ』の奥行きをさらに深めており,空と白「 」のこの言葉に込められた含蓄の深さを実感させる場面となっている.

\section{プログラムに関する著作権法10条3項の内容を調べて回答してください.
  同項に規定される「プログラミング言語」,「規約」,「開放」のそれぞれの用語の意味についても調べて回答してください.}%%%
プログラムの著作物を作成するために用いるプログラム言語,規約及び解法には,著作権による保護は及びません.

\textbf{プログラム言語}:プログラムを記述するための言語

\textbf{規約}:プログラム言語の文法や記述方法に関する取り決め

\textbf{解法}:プログラムによって解決しようとする問題の解決方法や手順
\section{パブリシティ権の侵害が問題となった事件を調べ,その概要を回答してください.}
パブリシティ権の侵害が問題となった代表的な事件として「ピンク・レディー事件」が挙げられる.
この事件では女性デュオ「ピンク・レディー」の写真が無断で雑誌の記事に使用されたことが争点となった.
具体的には,雑誌内でピンク・レディーの振り付けを利用したダイエット法が彼女たちの写真とともに紹介された.
これに対しピンク・レディー側は,写真の無断使用がパブリシティ権の侵害であるとして出版社を提訴した.
最終的に最高裁判所はパブリシティ権の存在を認めつつも,今回の写真使用は報道・評論の範囲内であり権利侵害には当たらないと判断した.
この判決は日本におけるパブリシティ権の解釈に大きな影響を与えた\cite{4_1}.

\section{芸人などの発する,いわゆる一発ギャグに著作権としての創作性が認められるかどうか,検討して回答してください.
  なお,この課題に言う一発ギャグとは,例えばIKKO氏の「どんだけ〜」,平野ノラ氏の「おったまげ〜/しもしも〜?」,
  小島よしお氏の「そんなの関係ねぇ」,といった,主に言葉が主体となるものを考えることとする}
一発ギャグが著作権の保護対象としての創作性を有するかは,独自の表現が認められるかどうかに大きく依存する.
著作権法上,「著作物」とは「思想または感情を創作的に表現したもの」を指し,一定の創作性が必要である\cite{5_1}\cite{5_2}.

たとえば,IKKO氏の「どんだけ〜」や小島よしお氏の「そんなの関係ねぇ」のようなフレーズは,一般的な言葉の組み合わせや短いフレーズであるため,
通常は著作権の保護が認められるほどの創作性を持たないとされる.
このため,単にフレーズ自体だけでは著作権の保護対象となる可能性は低い\cite{5_3}.

しかし,これらの一発ギャグが独自の動作や表情と組み合わさることで,他にはない一体的なパフォーマンスとして評価される場合,著作権が認められる可能性がある.
たとえば,「どんだけ〜」というフレーズは,IKKO氏の特定の手の動きや表情と組み合わせることで初めて彼らしい独自の表現となり,創作性が認められるケースがある.
このように,言葉だけではなく,動作や表情も含めた全体としての「表現」になることで,著作権の対象となることが考えられる\cite{5_4}.

また,著作権だけでなく,不正競争防止法や商標法に基づき保護されるケースもある.
特に,特定のフレーズや動作を商標登録している場合,無断で使用することは商標権侵害となる可能性がある.
たとえば,吉本興業所属の芸人がよく行うように,一発ギャグやキャッチフレーズが商標登録されると,これを第三者が無断で使用した場合には商標権を侵害することになる.
また,著名なフレーズや動作を無断使用し,他人が混同を生じさせるような場合,不正競争防止法の観点からも法的な問題が生じる\cite{5_5}\cite{5_6}.

以上のように,一発ギャグが著作権の保護対象となるかは単なる言葉の表現ではなく,独自の動作や表情といった要素が加わり創作的な「表現の一体性」が認められるかが重要である.
また,著作権だけでなく商標権や不正競争防止法など,複数の法律によっても保護される可能性があるため個別の表現内容や使用状況に応じた判断が必要である.
\section{いわゆるプリクラ機で作成した写真は創作性が認められるかを検討し,理由を付して回答してください.}
プリクラ機で作成された写真に創作性が認められるかを検討する.
著作権法上,写真が著作物として保護されるためには撮影者や加工を行った者の個性や創意が表現されていることが求められる.
一般的な写真では,構図やアングル,照明,被写体の配置などを撮影者が自由に決定できるため,その中に独自の意図や工夫が反映されやすい\cite{5_1}.

しかし,プリクラ機での撮影は利用者が構図やアングルを調整する自由が限られており,背景やスタンプ,フィルターなどのデザインも事前に機械側で提供されるテンプレートから選択する形式となっている.
このため,テンプレートを選ぶだけの操作では利用者の個性が十分に表現されていないとみなされ,著作物としての創作性が認められにくいと考えられる.

一方で,利用者が撮影後に手書きの文字やイラスト,独自のスタンプ配置などで写真に装飾を施すことで個性的な要素が加わる場合がある.
たとえば,手書きのメッセージやオリジナルの配置が加わることで,そこには利用者の意図や感情が反映され創作性が認められる可能性が高くなる.
したがって,プリクラ機で作成された写真全体ではなく,利用者が追加した部分に関しては著作権法上の保護対象とみなされることがある.

結論として,プリクラ機での写真は通常機械によるテンプレートの使用が主体であるため創作性が認められにくいが,
利用者が手書きや独自の装飾を加えることでその装飾部分に限って著作物としての保護が認められる可能性がある.

\newpage
% 参考文献
\begin{thebibliography}{99}
  \bibitem{R6}令和6年度著作権テキスト\url{https://www.bunka.go.jp/seisaku/chosakuken/textbook/index.html}
  \bibitem{no_game_no_life}ノーゲーム・ノーライフ 3巻
  \bibitem{4_1}ピンク・レディー事件最高裁判決 ~著名人の写真利用とパブリシティ権を考える 鈴木里佳|コラム | 骨董通り法律事務所 For the Arts\url{https://www.kottolaw.com/column/000371.html}
  \bibitem{5_1}著作隣接権 - これだけ知っとけ著作権講座\\\url{https://chosakuken-kouza.com/kihon/rinsetsuken.html}
  \bibitem{5_2}「実演家」とは?俳優やアーティストのこと…俳優やアーティストの権利は?「著作隣接権」 | 著作権&著作隣接権まとめノート…ときどきエンタメ日誌\\
  \url{https://learning-artistsrights.com/perfomer-neighboringrights/}
  \bibitem{5_3}変名 - Wikipedia\url{https://ja.wikipedia.org/wiki/%E5%A4%89%E5%90%8D}
  \bibitem{5_4}実演家とは - 芸団協CPRA 実演家著作隣接権センター\url{https://www.cpra.jp/performers/about/}
  \bibitem{5_5}実演家の著作隣接権の『実演家人格権』とは? | はじまりの部屋\url{https://keepgo5.com/what-is-the-performing-artists-personality-right/}
  \bibitem{5_6}実演家の権利 | 著作権のネタ帳\url{https://copyright-topics.jp/basic/jitsuenka_right/}
\end{thebibliography}

\end{document}