\documentclass[titlepage,a4paper]{jsarticle}
\usepackage{../../../sty/import}% 各種パッケージインポート
\usepackage{../../../sty/title}% タイトルページの変更

%% タイトルページの変数
% レポートタイトル
\title{社会福祉概論 第2回課題}
% 提出日
\expdate{\today}
% 科目名
\subject{社会福祉概論}
% 分野
\class{情報経営システム工学分野}
% 学年
\grade{B3}
% 学籍番号
\mynumber{24336488}
% 記述者
\author{本間三暉}
%
\begin{document}
% titleページ作成
\maketitle
% a,b合わせて800文字程度
\section{社会福祉協議会の根拠法令と内容}
社会福祉協議会は,地域社会における福祉の推進と支援を目的とした公益性の高い団体である.
その根拠法令は主に社会福祉法であり,第109条に基づき地域福祉活動を担う主体としての役割を果たすことが明記されている.
加えて介護保険法も根拠法令の一つとなっており,地域包括支援センターの運営や介護予防事業など,高齢者の支援や生活の質向上に資する活動も担うことが可能である.

社会福祉協議会の具体的な活動内容には地域福祉計画の策定が含まれる.
これは地域ごとの福祉ニーズに基づき福祉活動の方向性を示すものである.
また,ボランティア活動の推進にも力を入れ,地域住民や団体と協力して助け合いの仕組みを形成する.
さらに,高齢者,障害者,子育て家庭への福祉サービスの提供や相談支援,生活支援サービスなどを行い日常生活に必要な支援を提供する.
災害時には被災者への緊急支援や避難所の運営も担い,地域の安心安全に寄与している.
地域住民への福祉啓発活動も行い,福祉に関する知識や意識を深めることで地域全体の福祉意識の向上を図っている.

これらの活動を通じて社会福祉協議会は地域の福祉環境の向上と住民の生活の質向上に寄与しているのである.
\section{長岡社会福祉協議会の活動内容}
長岡社会福祉協議会は,新潟県長岡市において地域福祉の推進を目指す団体である.
その活動は地域のニーズに応じた多様な福祉支援に焦点を当てている.
主な活動としてまず地域福祉活動計画の策定が挙げられる.
これは,地域住民が安心して暮らせる環境を整えるための方針を示すものである.
また,高齢者や障害者,子育て家庭への支援活動も行っており,生活支援サービスや相談支援が提供されている.

さらに,長岡社会福祉協議会はボランティア活動の推進にも力を入れている.
市民や各団体との協力により,地域の助け合いの輪を広げ災害時には避難所運営や緊急支援活動も行う.
また,地域住民への福祉啓発活動を通じて,福祉の理解を深め,支え合いの文化を醸成する役割も担っている.
これにより,長岡市における住民の生活の質向上と地域福祉の推進に寄与しているのである.

% % 参考文献
\begin{thebibliography}{99}
  \bibitem{a}社会福祉法(e-Gov法令検索)\\\url{https://elaws.e-gov.go.jp/document?lawid=326AC0000000045}
  \bibitem{b}介護保険法(e-Gov法令検索)\\\url{https://elaws.e-gov.go.jp/document?lawid=409AC0000000123}
  \bibitem{c}全国社会福祉協議会 公式ウェブサイト\url{https://www.shakyo.or.jp/}
  \bibitem{d}長岡市社会福祉協議会\url{http://www.nagaoka-shakyo.or.jp/}
\end{thebibliography}

\end{document}