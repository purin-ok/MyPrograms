\documentclass[titlepage,a4paper]{jsarticle}
\usepackage{../../../sty/import}% 各種パッケージインポート
\usepackage{../../../sty/title}% タイトルページの変更

%% タイトルページの変数
% レポートタイトル
\title{社会福祉概論 第4回課題}
% 提出日
\expdate{2024年11月26日}
% 科目名
\subject{社会福祉概論}
% 分野
\class{情報経営システム工学分野}
% 学年
\grade{B3}
% 学籍番号
\mynumber{24336488}
% 記述者
\author{本間三暉}
%
\begin{document}
% titleページ作成
\maketitle
最後のセーフティネットである生活保護制度における自立助長機能の強化とともに,生活保護受給者以外の生活困難者に対する,いわゆる「第2のセーフティネット」の充実・強化を図ることを目的として,平成27(2015)年4月から生活困窮者自立支援制度が施行されている.

一方,民間支援団体として「フードバンク」は,食品ロスを福祉に役立てる,もっともわかりやすい形である活動を行っている.
ついては,「フードバンク」の活動内容や機能性をまとめなさい.
\section{活動内容}
フードバンク活動は,食品ロスを削減し,福祉の向上に寄与することを目的とした取り組みである.
具体的には,企業や個人から寄付された余剰食品を回収し,生活に困窮している人々や福祉施設に無償で提供する仕組みである.
これにより,食品の有効活用と生活困窮者支援を同時に実現している.

\section{フードドライブ}
フードドライブとは,地域や企業,学校などで食品を募る活動を指す.
缶詰や乾燥食品,賞味期限が近いが安全に食べられるものなどを広く集め,フードバンクを通じて必要とする人々に提供する.
この活動は,地域社会の意識向上や食品ロス問題への関心を高めるきっかけにもなっている.

\section{フードバンク ながおか}
フードバンクながおかは,新潟県長岡市を拠点に,食品ロス削減と生活困窮者支援を目的とした活動を展開している.2014年に「フードバンクにいがた長岡センター」として設立され,2021年4月からは地域に密着した活動を強化するため,「フードバンクながおか」として再編成された.
\subsection{活動目的}
フードバンクながおかの活動目的は,地域における食品ロスの削減と生活困窮者支援を通じて,誰もが食を分かち合える持続可能な社会を実現することである.具体的には,企業や市民から寄贈された食品を有効活用し,ひとり親家庭や生活困窮者,福祉施設などの支援を必要とする方々に提供することで,生活の安定を図ることを目指している.
\subsection{活動内容}
フードバンクながおかの活動内容としては,まず企業や個人から賞味期限が十分に残っており,常温保存が可能な食品を寄贈してもらうことが挙げられる.また,地域のイベントや企業内でフードドライブを実施し,広く食品寄贈を募る取り組みも行っている.集められた食品は仕分けや袋詰めを行い,その後,ひとり親家庭や生活困窮者,福祉施設,子ども食堂などへ提供される.これらの作業は多くのボランティアの協力を得て進められており,地域社会全体で支援の輪を広げている.
\end{document}