\documentclass[titlepage,a4paper]{jsarticle}
\usepackage{import}% 各種パッケージインポート
\usepackage{title}% タイトルページの変更

%% タイトルページの変数
% レポートタイトル
\title{特別課題(なれる!SE)}
% 提出日
\expdate{\today}
% 科目名
\subject{情報システム設計論}
% 分野
\class{情報経営システム工学分野}
% 学年
\grade{B3}
% 学籍番号
\mynumber{24336488}
% 記述者
\author{本間三暉}

\begin{document}
% titleページ作成
\maketitle
\section{1巻}
\subsection{内容の要約}

\subsection{得られたネットワーク設計上の教訓}

\section{2巻}
\subsection{内容の要約}

\subsection{得られたネットワーク設計上の教訓}

\section{3巻}
\subsection{内容の要約}

\subsection{得られたネットワーク設計上の教訓}

\section{4巻}
\subsection{内容の要約}

\subsection{得られたネットワーク設計上の教訓}


% 参考文献
\begin{thebibliography}{99}
\bibitem{SE}なれる!SE|1~16巻 \url{https://kakuyomu.jp/works/1177354054886136854}
\end{thebibliography}

\end{document}