\documentclass[titlepage,a4paper]{jsarticle}
\usepackage{../../../sty/import}% 各種パッケージインポート
\usepackage{../../../sty/title_team}% タイトルページの変更

%% タイトルページの変数
% レポートタイトル
\title{}
% 提出日
\expdate{\today}
% 科目名
\subject{産学連携実践的AI応用}
% 分野
\class{情報経営システム工学分野}
% 学年
\grade{B3}
% 学籍番号
\mynumber{24336488}
% 記述者
\author{本間 三暉}
% グループ名 % もし班があるやつならtitle_team.styを入れる
\team{3}
% 共同実験者 % もし共同実験者が必要なやつならtitle_kyoudou.styを入れる
\coauthor{%
\textbf{学籍番号:22100986} & \textbf{氏名:板山 修大}\\
\textbf{学籍番号:22105590} & \textbf{氏名:筒井 翼}\\
\textbf{学籍番号:22106685} & \textbf{氏名:花田 光}\\
\textbf{学籍番号:22100289} & \textbf{氏名:浅野 繭}\\
}
% 記載例:
%\coauthor{%
% 学籍番号:24567321 & 氏名:吉田 富美男 \
% 学籍番号:12345678 & 氏名:安藤 雅洋 \
% 学籍番号:13579234 & 氏名:雲居 玄道 \
%%

\begin{document}
% titleページ作成
\maketitle
\section{データの観測構造をモデル化する}
\section{解くべき問題を特定する}

\section{観測データのみを用いて問題を解く方法を考える}
\section{機械学習モデルを学習する or 統計分析手法を適用する}
\section{施策を導入する}
\section{施策導入後の分析}
\section{他のグループ発表を踏まえた比較と考察}
\section{まとめと今後の課題}
\section{授業に関する感想と要望}
% 参考文献
\begin{thebibliography}{99}
  \bibitem{}
\end{thebibliography}

\end{document}