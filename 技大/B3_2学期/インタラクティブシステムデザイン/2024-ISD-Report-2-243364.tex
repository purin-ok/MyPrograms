\documentclass[titlepage,a4paper]{jsarticle}
\usepackage{../../sty/import}% 各種パッケージインポート
\usepackage{../../sty/title}% タイトルページの変更

%% タイトルページの変数
% レポートタイトル
\title{インタラクティブ・システム・デザイン期末レポート(課題2)}
% 提出日
\expdate{\today}
% 科目名
\subject{インタラクティブ・システム・デザイン}
% 分野
\class{情報経営システム工学分野}
% 学年
\grade{B3}
% 学籍番号
\mynumber{24336488}
% 記述者
\author{本間三暉}

\begin{document}
% titleページ作成
\maketitle
\begin{center}
  インタラクティブ・システム・デザイン期末レポート(課題2)
\end{center}
\begin{flushright}
  \today

  24336488,本間三暉
  \vskip\baselineskip % 一行空行
\end{flushright}

\section{単文による問題記述}
\renewcommand{\thesubsection}{\thesection)}
\subsection{ }
出張者も利用可能なオンライン会議管理システム(22文字)

\section{各要素の説明}
\renewcommand{\thesubsection}{\thesection-\arabic{subsection})}
\subsection{支援されるアクティビティ(何を)}
50人の専門家のグループがミーティングのスケジュールを管理し,効率的に調整する.(40文字)
\subsection{ユーザ(誰を)}
出張中も含むグループ内のメンバーおよび管理者が主なユーザである.管理者は特に効率的な運営が求められる.(51文字)
\subsection{支援のレベル(どの程度)}
手作業の壁面カレンダーに代わり,リアルタイムでの会議予約や変更通知を自動化し,時間ロスを削減する.(49文字)
\subsection{解の形式(どのように)}
オンラインのカレンダーシステムで,場所や会議時間の予約,変更通知をクラウド経由で提供する.(45文字)
\section{考察}
\renewcommand{\thesubsection}{\thesection)}
\subsection{}
「グループ・オンライン・カレンダー」は,50人の専門家が行う内部および顧客とのミーティング管理に必要なシステムである.
現在の手作業による壁面カレンダーの使用は,出張中のメンバーにはアクセスできず,管理者が変更を電話連絡するなど非効率的である.
このシステムの導入により,グループ全体の予定がオンラインで共有され,出張中でも即時に確認できる.
さらに,変更が自動通知されることで管理者の負担が軽減される.

問題定義の重要な側面として,ユーザは出張者や管理者を含む50名のグループであり,支援のレベルはリアルタイム更新と通知自動化が中心となる.
特に会議の重複や時間超過による再調整の問題に対して,効率的な解決手段となる.
解の形式として,クラウドベースのオンラインカレンダーを活用し,メンバーがいつでもどこでもアクセスできることが必須条件である.

このシステム導入により,時間的ロスを最小化し,業務効率の向上が期待される.
また,デジタル化による記録管理の信頼性向上と,情報の共有・修正の簡便化も大きな利点である.(455文字)

% 参考文献
\begin{thebibliography}{99}
  \bibitem{index}インタラクティブ・システム・デザイン資料
\end{thebibliography}

\end{document}