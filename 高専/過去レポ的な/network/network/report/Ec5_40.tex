\documentclass{jsarticle}
\usepackage[dvipdfmx]{graphicx}
\usepackage{bm}
\usepackage{amsmath}
\usepackage{amssymb}
\usepackage{amsfonts}
\usepackage{comment}
\usepackage{listings}
\usepackage{cases}
\usepackage{siunitx}
\usepackage[hyphens]{url}
\usepackage{listings}
\usepackage{jlisting}
\lstset{
    basicstyle={\ttfamily},
    identifierstyle={\small},
    commentstyle={\smallitshape},
    keywordstyle={\small\bfseries},
    ndkeywordstyle={\small},
    stringstyle={\small\ttfamily},
    frame={tb},
    breaklines=true,
    columns=[l]{fullflexible},
    numbers=left,
    xrightmargin=0zw,
    xleftmargin=3zw,
    numberstyle={\scriptsize},
    stepnumber=1,
    numbersep=1zw,
    lineskip=-0.5ex,
    keepspaces=true,
    language=c
}
\renewcommand{\lstlistingname}{リスト}
\makeatletter
\newcommand{\figcaption}[1]{\def\@captype{figure}\caption{#1}}
\newcommand{\tblcaption}[1]{\def\@captype{table}\caption{#1}}
\makeatother

\title{\vspace{-3cm}ネットワークプログラミング レポート}
\author{Ec5 40番 若月 耕紀}
\date{}

\begin{document}
\maketitle

\section{はじめに}
本レポートでは,ネットワークプログラミングのロボットプログラムについての動作を記す.

\section{プログラムの動作}
作成したロボットプログラムの動作について以下に示す.

\subsection{自身やエネルギータンクの情報取得}

目標とするエネルギータンクを決めるために,自身の位置情報と,フィールドのエネルギータンクの情報を取得する.取得する情報は,自身の座標,各エネルギータンクの座標と得点とする.自身の情報を取得するコードとエネルギータンクの情報を取得するコードをリスト1,2に示す.

\begin{lstlisting}[caption=自身の情報を取得するコード(一部抜粋)]
String[] str_ship = line.split(" ");
if(str_ship[0].equals(name)){
	ship_x = Integer.parseInt(str_ship[1]);
	ship_y = Integer.parseInt(str_ship[2]);
}
\end{lstlisting}

\begin{lstlisting}[caption=エネルギータンクの情報を取得するコード(一部抜粋)]
String[] str_energy = line.split(" ");
energyX[energyVal] = Integer.parseInt(str_energy[0]);
energyY[energyVal] = Integer.parseInt(str_energy[1]);
energyP[energyVal] = Integer.parseInt(str_energy[2]);
\end{lstlisting}

\subsection{目標の決定}

今回作成したプログラムでは,基本的には自身に最も近いエネルギータンクを目標とする.ただし,2点間の距離を算出する際に,エネルギータンクのポイントに応じて距離を短くするように減算を行う.また,境界を越えると反対側に出ることを考慮して計算する.

目標とするエネルギータンクの座標を取得するコードをリスト3に示す.

\begin{lstlisting}[caption=目標とするエネルギータンクの座標を取得するコード(一部抜粋)]
if(energyVal > 1){
	x = 128 - Math.abs(128 - Math.abs(ship_x - energyX[0]));
	y = 128 - Math.abs(128 - Math.abs(ship_y - energyY[0]));
	r1 = (x * x) + (y * y) - (energyP[0] * energyP[0] * weight);
	for(i = 1, j = 0; i < energyVal; i++){
		x = 128 - Math.abs(128 - Math.abs(ship_x - energyX[i]));
		y = 128 - Math.abs(128 - Math.abs(ship_y - energyY[i]));
		r2 = (x * x) + (y * y) - (energyP[i] * energyP[i] * weight);
		if(r1 > r2){
			r1 = r2;
			j = i;
		}
	}
}else{
	j = 0;
}
TargetEnergyX = energyX[j];
TargetEnergyY = energyY[j];
\end{lstlisting}

\subsection{移動について}

自身の座標とエネルギータンクの座標を長方形(正方形)の対角の頂点として考える.この時,長方形に近い形であれば,上下または左右に大きく移動させ,正方形に近い形であれば,斜めに移動させる.

移動する方向を決める際には,目標を決定する際と同様に,境界を越えたら反対側に出ることを考慮する.移動方向を決める条件分岐のコードをリスト4に示す.

\begin{lstlisting}[caption=移動方向を決める条件分岐のコード(一部抜粋)]
x = ship_x - TargetEnergyX;
y = ship_y - TargetEnergyY;
if(Math.abs(Math.abs(x) - Math.abs(y)) >= shapeError){/長方形に近い場合
	if((Math.abs(x) - Math.abs(y)) >= 0){/横に長い場合
		if(Math.abs(x) <= 128){/順方向
			if((x) <= 0){/右側
		}
	}else{/縦に長い場合
		if(Math.abs(y) <= 128){/順方向
			if((y) <= 0){/上側
	}
}else{/正方形に近い場合
	/左右判定
	if(Math.abs(x) <= 128){/順方向
		if((x) <= 0){/右側
	/上下判定
	if(Math.abs(y) <= 128){/順方向
		if((y) <= 0){/上側
}
\end{lstlisting}

\end{document}