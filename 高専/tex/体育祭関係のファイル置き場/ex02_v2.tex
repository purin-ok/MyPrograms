\documentclass[titlepage]{jarticle}
\usepackage{r03ec-exp}
\usepackage[dvipdfmx]{graphicx}
%
%%% 表紙の記載事項設定
%
% 実験題目  ※レポートを書くときは,まず,タイトルを正しいものに変えましょう
%
\title{LED点滅回路の製作}
% 学年・番号
\grade{}%
% 氏名
\author{}
% 班(後期は班に分かれて実験をする.そのときは,ここに班番号を記入する.)
\team{}
% 提出日
\date{2021年5月31日}
% 実験日
\expdate{2021年5月17日, 5月24日, 5月31日}
% 共同実験者
% グループに分かれて実験をするテーマでは,グループメンバーの番号名前を書く.
\coauthor{}
%
%記載例:基本練習
%\coauthor{%
%  2番 & 新潟 花子\\
%  11番 & 三条 次郎}
%%

\begin{document}
\maketitle

\section{目的}
\begin{itemize}
  \item 電子部品に関する基礎知識や取り扱い方法を学ぶ。
  \item ハンダゴテやニッパなど工具の正しい使い方を再確認する。
  \item 簡単な電子回路の動作原理を理解する。
  \item 回路図を元に基盤上での部品のレイアウトや実体配線を考える。
\end{itemize}

\section{動作原理}

\section{使用器具}
	\begin{itemize}
		\item はんだごて
		\item はんだ
		\item ニッパ
		\item ラジオペンチ
	\end{itemize}

\section{実験方法}
  \subsection{実体配線図の作成}

  \subsection{組み立て}

  \subsection{動作確認}

\section{考察}

\section{感想}

\end{document}