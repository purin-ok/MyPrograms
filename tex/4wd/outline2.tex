% \documentclass[twoside,twocolumn]{ujarticle}
\documentclass[titlepage]{jarticle}
%\usepackage{type1cm}
\usepackage{outline-ec}
\usepackage{amsmath,amssymb,verbatim,ascmac,multicol}
\usepackage{tabularx}
% dvioutで確認する場合は以下を有効にする
%\usepackage[dviout]{graphicx,color}
% pdf化する場合は以下を有効にする
\usepackage[dvipdfmx]{graphicx,color}
%
% --------------------------------------------------------------------------
% 図表番号の後の:を削除
%
\makeatletter
\long\def\@makecaption#1#2{% #1=図表番号,#2=キャプション本文
\sbox\@tempboxa{#1 \hskip0.5zw #2}% 図表番号とキャプションの間のスペース 0.5zw
\ifdim \wd\@tempboxa >\hsize
#1 #2\par 
\else
\hb@xt@\hsize{\hfil\box\@tempboxa\hfil}
\fi}
\makeatother
% --------------------------------------------------------------------------

%
% --------------------------------------------------------------------------
% 下の該当する部分を書き換える
%
\氏名{本間 三暉}			%% 自分の氏名
\出席番号{35}					%% 出席番号
\研究室名{視覚情報処理研究室}			%% 研究室名
\指導教官{高橋 章}			%% 指導教員名
%
% --------------------------------------------------------------------------
% 研究題目等
%
% 研究題目が2行になるときは \\ で行送りできる
%
\発表番号{B\;--\;2}
\研究題目{単一視点による人物の全身運動の三次元計測について}

% --------------------------------------------------------------------------
% アブストラクト
%
\アブストラクト{
全身運動を行う人物を一台のRGBカメラやRGBDカメラで撮影し,それぞれのカメラに合った画像処理を用いた手法で三次元骨格推定を行う.
また,それ以外にも一台のカメラと画像処理を用いて三次元骨格推定を行う方法がないかリサーチする.
そうして取得した三次元骨格推定の結果を精度や処理速度などについて比較し分析する.
% また,組み込みPCで実装する場合やカメラに対し手前にある物体が後ろにある物体を隠す状態になり十分に計測できないオクルージョンへの対応,
% リアルタイム処理を目的とした高速化を目指す.
% =======
% モーションキャプチャデバイスmocopiでは表現しきれない部分を画像処理を用いて補助することを検討する.
% mocopiでは関節部などが正確に計測できないことがあるため,竹馬や一輪車などの精密な重心移動や姿勢制御が必要な運動に関して測定を行い,
% 画像処理による骨格推定を組み合わせ,バランス運動時の重心や姿勢を解析することによって,mocopiの測定精度を上げることを目指す.
}
%
% --------------------------------------------------------------------------
% 本文開始
%
\begin{document}
\maketitle

%
% --------------------------------------------------------------------------
% 第1節
%
\section{研究背景・目的}
%
情報通信技術の急速な進歩により人工現実感,拡張現実感,複合現実感などの応用が広がっている.
感染症対策を契機にオンラインコミュニケーションも増加し,インターネット上の仮想共有空間であるメタバースが注目されている\cite{meta}.

その理由の一つに,離れている相手にテキストや音声だけでなく身振りや動作などのノンバーバルな情報の伝達を行うことが容易であるという点がある.
三次元の仮想空間で自分の分身となるアバターを自由に操作するには,
体の動きを計測する必要があり,画像処理による方法\cite{CV}や専用デバイスを装着する方法\cite{キャプチャ}などが試みられている.
画像処理による方法で三次元の情報を取得するためには先行研究のような複数台のカメラを用いる方法\cite{turugi}があるが,狭い室内であるなどの場所の制約や,
限られた予算の中で実装したいという資金の制約などによって,複数台のカメラを用いる方法取るのが難しい場合がある.

本研究では1台の入力装置で三次元骨格推定ができる現行の方法について調査及び実装を行い,それぞれの方法の計測速度や精度ついて比較する.

% 情報通信技術の急速な進歩により人工現実感,拡張
% 現実感,複合現実感などの応用が広がっている.感染
% 症対策を契機にオンラインコミュニケーションも増加
% し,インターネット上の仮想共有空間であるメタバー
% スが注目されている.メタバースが注目されている理
% 由の一つに,離れている相手にテキストや音声だけで
% なく身振りや動作などのノンバーバルな情報の伝達
% を行うことが容易であるという点がある.三次元の仮
% 想空間で自分の分身となるアバターを自由に操作する
% には,体の動きを計測する必要があり,画像処理によ
% る方法 [1] や専用デバイスを装着する方法 [2] などが試
% みられている.画像処理による方法で三次元の情報を
% 取得するためには先行研究のような複数台のカメラを
% 用いる方法 [3] があるが,狭い室内であるなどの場所
% の制約や,限られた予算の中で実装したいという資金
% の制約などによって,複数台のカメラを用いる方法や
% 専用デバイスを装着する方法を取るのが難しい場合が
% ある.
% 本研究ではカメラ 1 台で三次元骨格推定ができる
% 現行の方法について実験を行い,それぞれの方法のメ
% リットやデメリット,精度などについて比較する.ま
% た,それらを元に組み込み PC での実装やリアルタイ
% ム処理などの高速化,オクルージョンというカメラに
% 対し手前にある物体が後ろにある物体を隠す状態にな
% り十分に計測できない場合への対応を目指す.

% 本研究では市販のモーションキャプチャデバイスmocopiを,運動解析に活用することを検討する.
% このデバイスは両手,両足,頭,腰の計6か所に小型センサを装着してリアルタイムに三次元計測を行うことができるが,肘や膝などの関節の屈曲を正確に計測することができない.
% そこで画像処理による骨格推定を組み合わせ,一輪車や竹馬のような器具を使うバランス運動の動作解析を実現する.
% これにより,身体が身体自信や使用器具などに隠れてしまう場合の骨格推定の精度低下の解消,三次元計測の精度の向上,計測速度のさらなる改善などの問題を解決することが出来る.
% そして、スポーツや映像作品などの様々な分野で,使い方が限定されていたモーションキャプチャの応用範囲が広がることが期待できる.
%
% --------------------------------------------------------------------------
% 第2節
%
\section{研究内容}
%これって研究背景に描いたほうが自然か?

%
% --------------------------------------------------------------------------
% 第2節 第1小節
%
\subsection{人の動作の計測方法}
%
人の動作の三次元計測を行うには,画像処理による方法やモーションセンサによる方法などがある.それぞれの方法について簡単にまとめたものを表\ref{3D_1}に示す.
画像処理による方法では画像から人の骨格を推定することで人の動作を解析できる.
画像処理による三次元骨格推定は撮影するカメラに,色情報を記録できる一般的なRGBカメラを用いて解析する方法と,
カメラと物体の距離を測ることができるRGBDカメラ用いて解析する方法がある.
% これらの方法はモーションセンサを用いるものと比べると精度にばらつきが生まれる.

%% モーションセンサのしゅるいについてもかいたほうが自然か?
% モーションセンサによる方法は光学式や慣性式などがあるが,どの方法も精度が高いかわりにマーカーやセンサを検出対象に取り付けなければならないため使用できる環境が限定されてしまう.
モーションセンサによる方法は,光学式や慣性式などがある.
光学式は体表面にマーカーを取り付けそのマーカーを複数台のカメラで取り込むことで骨格を推定する.
慣性式は加速度,角速度,方位を測定できるセンサを体表面の指定箇所に取り付けることで骨格を推定する.

\begin{table}[t!]
  \centering
  \caption{動作を計測する方法の種類と特徴}
  \begin{tabular}{l|ll|ll}
    \hline
                  & \multicolumn{2}{c|}{\small{カメラ}}       & \multicolumn{2}{c}{\small{モーションセンサ}}                                                  \\ \cline{2-5}
                  & \multicolumn{1}{c|}{\small{RGB}}       & \small{RGBD}                         & \multicolumn{1}{c|}{\small{光学式}} & \small{慣性式} \\ \hline
    \small{センサ装着} & \multicolumn{1}{c|}{\small{不要}}        & \small{不要}                           & \multicolumn{1}{c|}{\small{必要}}  & \small{必要}  \\
    \small{外から撮影} & \multicolumn{1}{c|}{\small{必要}}        & \small{必要}                           & \multicolumn{1}{c|}{\small{必要}}  & \small{不要}  \\
    \small{必要台数}  & \multicolumn{1}{c|}{\small{1$\sim$数台}} & \small{1台}                           & \multicolumn{1}{c|}{\small{複数台}} & \small{0台}  \\ \hline
  \end{tabular}
  \label{3D_1}
\end{table}

% 動作人の動作の三次元計測を行うには,画像処理に
% よる方法やモーションセンサによる方法などがある.
% 画像処理による方法では画像から人の骨格を推定する
% ことで人の動作を解析することができる.画像処理に
% よって三次元骨格推定するには,撮影するカメラに色
% 情報を記録する一般的な RGB カメラで撮影して解析
% する方法と,kinect や RealSense のようなカメラと物
% 体の距離を測ることができる RGBD カメラで撮影し
% て解析する方法がある.モーションセンサによる方法
% は光学式や慣性式などがあるが,どの方法もマーカー
% やセンサを検出対象に取り付けなければならないため
% 使用できる環境が限定されてしまう.
% 本研究では場所や資金の制約があるような場面に対
% 応できるように,一台のカメラと画像処理により三次
% 元計測を行う場合の方法について検証する.

本研究では一台のカメラと画像処理で三次元計測を行う複数の方法について検証し,モーションセンサを用いて三次元計測した場合と比較する.
% 今回は場所や資金などの制約から一台のカメラで人の動作の三次元計測を行う場合に限定して比較するので,
% 先行研究\cite{turugi}のような複数台のカメラを用いて三次元計測を行う場合を除いた方法について検証する.

%
% --------------------------------------------------------------------------
% 第2節 第2小節
%
\subsection{RGBカメラ一台を用いた三次元骨格推定}
%
RGBカメラで撮影した動画をMediaPipe Pose\cite{cubemos}というライブラリを用いて処理することで骨格推定を行う.

MediaPipeはGoogleが提供しているライブメディアやストリーミングメディア向けの機械学習ソリューションである.
特徴として,少ない記述量で使うことができて,計算コストが低くスマートフォンや組み込みPCなど資源の限られたハードウェアでも使用できるというものが挙げられる.
また,その中でもMediaPipe Poseは動画から人間の姿勢を推論するライブラリで,動画から図\ref{RGB}に示す全身の33個のランドマーク位置または上半身の25個のランドマーク位置を予測できる.

\begin{figure}[b!]
  \centering
  \includegraphics[width=7cm]{img/media.png}
  \caption{MediaPipe Poseで取得できるランドマーク}
  \label{RGB}
\end{figure}

\subsection{RGBDカメラで行う三次元骨格推定}
入力を
% kinect2やintelのRealSenseなどの
RGBDカメラにする場合,kinectSDK\cite{kinectSDK}などのハードウェアの公式から出ているソフトウェアやmediapipe\cite{cubemos}などのオープンソースを処理に用いることによって三次元骨格推定を行うことができる.
RGBDカメラを入力装置として用いる際の関係を図\ref{app}に示す.
kinectSDK\cite{kinectSDK}のようなデバイス専用のソフトウェアや,Nuitrack\cite{cubemos}などのオープンソースを開発ライブラリに用いて処理することによってアプリケーションを作成できる.


RGBDカメラと開発ライブラリの組み合わせを表\ref{RGBD}に示す.


\begin{table}[t!]
  \centering
  \caption{RGBDカメラと開発ライブラリの組み合わせ}
  \begin{tabular}{l|l}\hline
    カメラ       & 開発ライブラリ          \\\hline
    kinect2   & kinectSDK,OpenNI \\
    RealSense & Nuitrack         \\\hline
  \end{tabular}
  \label{RGBD}
\end{table}

kinectSDKはMicrosoft Kinect for Windows Software Development Kit の略で,Microsoft社から公式にリリースされた開発キットである.
Windows上でkinect2を動かすのに必要なドライバやドキュメントなどが同梱されている.
kinectSDKはハードウェアであるkinect2との組み合わせに限定されており,現在は開発が終了してしまっている.

kinectSDKと似たような機能を持つもので,Kinectのセンサ部分を開発したPrimeSense社が中心となって開発したOpenNIというライブラリがある.
% OpenNIの構成を図\ref{app}に示す.
OpenNIには部分的なトラッキングやジェスチャ検出機能などkinectSDKに比べできることが多く,GitHub上でソースコードが公開されており現在も有志による開発が行われている.

\begin{figure}[b!]
  \centering
  \includegraphics[width=6cm]{img/app3.jpg}
  \caption{OpenNIの構成}
  \label{app}
\end{figure}


% % kinect2で撮影する場合はkinectSDKやOpenNIを用いて解析することで三次元骨格推定を行うことができるが,kinectSDKとOpenNIの両方に長所と短所があるためそれぞれの場合について実装し比較を行う.

NuitrackはRealSense D400シリーズやOrbbec Astraシリーズ等のRGBDカメラで骨格認識を可能にするミドルウェアで,
全身の19個の関節部分のトラッキングが可能である.
%
% --------------------------------------------------------------------------
% 第2節 第3小節
%
% \subsection{}
%

%
% --------------------------------------------------------------------------
% 第2節 第4小節
%
\subsection{画像処理データの比較方法}
%
各画像処理の精度を比較する際,画像処理による方法で取得したデータを基準にするのは特定の骨格推定の方法に有利な結果が出てしまう可能性があるため,
基準とするデータは画像処理に頼らない独立した方法で取得する必要があるなどの理由からモーションキャプチャデバイスmocopi\cite{mocopi}を用いる.

mocopiとは,市販のモーションキャプチャデバイスで両手,両足,頭,腰の計6か所に小型センサを装着してリアルタイムに三次元計測を行うことができる.
6つの小型センサで測定しているため肘や膝などの関節部の屈曲を正確に表現することはできないが,mocopiのセンサはそれぞれ3つの自由度を持つ加速度センサと角度センサで測定しており,
機械学習を用いることで,肘や膝などの関節部を含めた全身の推定を実現している.

% mocopiとは,市販のモーションキャプチャデバイスで両手,両足,頭,腰の計6か所に小型センサを装着してリアルタイムに三次元計測を行うことができる.
% mocopiのセンサはそれぞれ3つの自由度を持つ加速度センサと角度センサで測定しており,AIを利用して人の様々な動作を予め学習させておくことで,センサを装着していない肘や膝などの中間関節を含めた全身の推定を実現している.
% よってmocopiでは肘や膝などの関節の屈曲を正確に表現することはできなが

% 加速度センサと角度センサが搭載されているモーションキャプチャデバイスmocopiを用いて取得した骨格データとの誤差を元に精度を比較する.
% 理由として,現行のRGBカメラやRGBDカメラで撮影し処理を行うような方法を比較する際,画像処理による方法で取得したデータを基準にするのは特定の骨格推定の方法に有利な結果が出てしまう可能性があり,
% 基準とするデータは画像処理に頼らない独立した方法で行う必要があるためである.

本研究では,ラジオ体操のような全身運動をしている人物一人に対して計測を行い,
mocopiのセンサの位置に当たる両手,両足,頭,腰の計6か所に関して,画像処理を用いて行った三次元骨格推定で得られた座標との誤差や時間変動を比較することで精度を評価する.
また,計測する運動は動作の緩急とオクルージョンというカメラに対し手前にある物体が後ろにある物体を隠す状態になり十分に計測できない場合の有無についてそれぞれの動作を満たす物を抜粋する.
%
% --------------------------------------------------------------------------
% 第2節 第5小節
%
% \subsection{}
%

%
%
%
% --------------------------------------------------------------------------
% 第2節 第6小節
%
%\subsection{}
%

%
%
%
% --------------------------------------------------------------------------
% 第3節
%
\section{研究計画と進捗状況}
%
mocopiを装着してラジオ体操をしてる人をRealSenseで撮影し,そのデータを用いて骨格推定を行う.
推定した骨格とmocopiで測定した骨格の両手,両足,頭,腰の座標のズレを誤差として精度の計測を行う.


%
% --------------------------------------------------------------------------
% 第3節 第1小節
%
% \subsection{研究の進め方}
%
% mocopiを装着して学校体操をしている人をRGBカメラ,kinect2,RealSenseで同時に撮影する.
% それぞれのカメラで撮影した映像から三次元骨格推定を行い,推定した骨格とmocopiで測定した骨格の両手,両足,頭,腰の座標のズレを誤差として精度の計測を行う.

%
%
% --------------------------------------------------------------------------
% 第3節 第2小節
%
% \subsection{研究方法や装置の概略}
%
% ・撮影機材や開発環境について記述
% 本研究では,開発環境としてmocopiとOpenPoseを使用する.

% mocopi\cite{mocopi}とは,市販のモーションキャプチャデバイスで両手,両足,頭,腰の計6か所に小型センサを装着してリアルタイムに三次元計測を行うことができる.
% mocopiのセンサはそれぞれ3つの自由度を持つ加速度センサと角度センサ
% 現在は,OpenPoseによる姿勢推定を進めている.
% 今後はRGBカメラによる方法の他にRealSenseで撮影した動画から三次元骨格推定を行う測定プログラムをSDKやライブラリを用いて開発し,mocopiから得られた骨格データと比較分析する.
% その後,一台のカメラと画像処理で行える三次元骨格推定の方法についてリサーチしつつ逐次実装し,比較対象を増やす.
%
% ・mokopiに付いて記述
%


%
%
% --------------------------------------------------------------------------
% 第3節 第3小節
%
% \subsection{進捗状況}
%
% 現在は,OpenPoseによる姿勢推定を進めている.今後は記述した方法だけでなく,他にも単一のカメラで三次元骨格推定ができる方法がないかリサーチしつつ,オクルージョンへの対応,高速化,組み込みPCでの実装を目指していく.
%
% --------------------------------------------------------------------------
% 第3節 第4小節
%
%\subsection{}
%

%

%
%
% --------------------------------------------------------------------------
% 第3節
%
% \section{まとめと今後の予定}
%

% 学校体操や空手の型などの,体を大きく動かし,関節の位置や体の相対関係が正しく測定できるようにする.
%
% --------------------------------------------------------------------------
% 参考文献
%

\begin{thebibliography}{99}
  \small{
    \bibitem{meta}{
      日本経済新聞,``孤独感抱える若者の交流の場,メタバースに開設'',https://onl.la/Y7BehPP
    }
    \bibitem{CV}{
      平尾 公男ら,``多関節 CG モデルと距離画像による上半身の姿勢推定'',Technical report of IEICE.
      PRMU, VOL.104, No.573, 79-84, 2004.
    }
    \bibitem{キャプチャ}{
      白鳥 貴亮ら,``モーションキャプチャと音楽情報を用いた舞踊動作解析手法'',電子情報通信学会論文誌D,Vol.J88-D2,No.8,pp.1662-1671,2005.
    }
    \bibitem{turugi}{
      剱 一輝,``柔道競技の3Dアーカイブ化'',令和4年度長岡高専専攻科論文,2023.
    }
    % \bibitem{baseline}{
    %   J. Martinez, R. Hossain, J. Romero, J. Little. ``A simple
    %   yet effective baseline for 3d human pose estimation'' . In
    %   ICCV, 2017.
    % }
    \bibitem{ビデオ}{
      安達 康平ら,``ビデオからの3次元姿勢を用いた行動認識における精度向上の試み'',研究報告モバイルコンピューティングとパーベイシブシステム(MBL),
      2020-MBL-94,47,1-7,2020.
    }
    \bibitem{kinectSDK}{
      谷尻 豊寿,``体の動きがコントローラ C++でkinectプログラミング KINECTセンサ画像処理プログラミング'',株式会社 カットシステム,2011.
    }
    \bibitem{cubemos}{
      Google,``mediapipe'',https://developers.google.com/mediapipe
    }
    \bibitem{mocopi}{
      SONY,``モバイルモーションキャプチャーmocopi'',https://www.sony.jp/mocopi/
    }
    % \bibitem{openpose}{
    %   CAO,Zhe,et al.OpenPose: Realtime Multi-Person 2D Pose Estimation Using Part Affinity Fields. arXiv preprint arXiv:1812.08008. 2018.
    % }
  }
\end{thebibliography}
\rightline{URLは2023年10月4日にアクセス}
\end{document}
